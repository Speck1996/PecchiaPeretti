\section{Introduction}

\subsection{Introduction and Scope}

In this document the process of testing and validation activity of a Data4Help project made by an external team is presented.

In particular the project analyzed is the one that can be found at
https://github.com/Framonti/MontiNappi made by:
\begin{itemize}
\item[] Fabio Nappi: fabio.nappi@mail.polimi.it
\item[] Francesco Monti: francesco4.monti@mail.polimi.it
\end{itemize}

In order to have a better experience in analyzing the code behind the deliverable provided, the following documents that can be found in the link provided above have been read:
\begin{itemize}
\item[RASD]: this document was useful to get the overall functionalities that the team application should have provided.
\item[DD]: in this document a general description of the architecture adopted to build Data4Help is presented. In particular the part of the Architectural Design was very important to have an overall view of the source code.
\item[ITD]: this document was used to check what are the features implemented, as well as to have a little more specific view of the source code and testing done with respect to the DD.
\end{itemize}


In the following chapters the problem araised when setting up the system and testing it will be presented with the addition of some advice to improve it. 
Indeed the scope of the testing and validation was not to break the system, but rather recognizing its main weaknesses and understand how those weaknesses could be avoided or at least limited. 


\subsection{Acronyms}
\begin{itemize}
\item \textbf{DD}: Design Document
\item \textbf{ITD}: Implementation and testing document.
\item \textbf{RASD}: Requirements Analysis and Specifications Document
\end{itemize}



\subsection{Document structure}
The ADT document is structured in the following way:

\bigskip\noindent
Chapter 1 presents the main focuses of the document.

\bigskip\noindent
Chapter 2 describes the issues found during the deployment of the system.

\bigskip\noindent
Chapter 3 analyzes the system behavior, verifying that the requirement implemented were successfully achieved and describing the issues or bugs found during the testing process.

\bigskip\noindent
Chapter 4 covers other aspects such as the quality of code, expressing some appreciations on the functionalities and criticizing some general aspects of the repository delivered.

\bigskip\noindent
Chapter 5 contains the tables showing the effort spent by each testing team member, in testing and writing the document.

\bigskip\noindent
Chapter 6 contains the references of the document.




