\chapter{Specific requirements}

\section{External Interface Requirements}


\subsection{Hardware Interfaces}
Data4Help is a software application, therefore no hardware interface is needed. However the application needs a smartphone with working bluetooth and internet in order to guarantee its primary services (collecting and updating eHealth data from user devices). 
In addition, the AutomatedSOS service requires that the users' smartphone has location services enabled.


\subsection{Software Interfaces}
In order to guarantee a quick development Data4Help system will use the following external services:
\begin{itemize}
\item DataBase engine : this service will be used to manage all users data and satisfy third party queries
\item Map: individuals with AutomatedSOS service will be able to see the live position of their ambulance whenever a rescue procedure is activated.
\end{itemize}

\section{Scenarios}
\subsection{Scenario 1}

Rebecca has worked very hard in the last months and she has been a long stressful periods. Recently she had to skip several working days because she didn't feel very well. 
She consulted her doctor John who didn't notice anything alarming.
John recently discovered Data4Help and he thinks that it could be useful to monitor the health status of Rebecca on daily basis for the next weeks.
Then, John asked Rebecca to register to Data4Help and to use her smartwatch to collect some data.
Rebecca immediately registered to the service and accepted the request that John sent to her.
Now John can have a better look at his patient's health status and can make better diagnosis.


\subsection{Scenario 2}
Walter is a 75 years old man who lives alone in a small town.
His doctor has been monitoring him thanks to Data4Help because he has a cardiac disease.
Last week, Walter was at the supermarket and suddenly he fainted. Luckily, the store was crowded and therefore someone helped him and called an ambulance.
He is afraid it could happen again and because he lives alone, he decides to activate the AutomatedSOS service.


\subsection{Scenario 3}
FatBit is a startup that wants to launch its first wearable. To optimize their limited resources and to improve their ads, Fatbit's managers want to know wich are the countries with healtier people,wich one have the more active, and the average age of their potential consumers. Collecting this data it's not easy, especially when joining the market for the first time. Fortunately a new service, Data4Help, has just launched and one of its main features it's the distribution of eHealth data, organized according to different parameters, included the ones FatBit managers are interested in (age,location,steps taken daily).
Therefore FatBit's managers decide to register with a unique account as third-party, and start their data analysis thanks to Data4Help.


\subsection{Scenario 4}
Wile is an engineering student who wants to get back in shape. He decides to join the gym and to buy a wearable to see its progress over time. Initially Wile is satisfied by his wearable and the built in application but later he starts noticing that more and more wearables with improved design and many new features are launching in the market. Wisely, he notice that the built in application collects the data only for the producer's devices and so if he wanted to change his wearable without losing data he were pratically forced to buy a new watch from the same company. Therefore he decides to use Data4Help, allowing him to change his wearable with the one he likes the most without losing data.

\section{Functional Requirements}

\begin{enumerate}

%\item[\textbf{[G1]}] \textbf{Individuals eHealth data is correctly gathered from their devices and uploaded to the system.}

\item[\textbf{[G\ref{g:Data}]}] \textbf{\goalData}

\begin{enumerate}

\item[D1] Location and eHealth data are provided by individuals' devices and assumed to be correct.
\item[R1] Data4Help is able to store the data provided by individuals.
\item[R2] Users are able to login with the username and the password associated to their account.
\item[R3] Data4Help is able to pair with the user wearable.
\item[R4] Individuals can upload their data through Data4Help app.
\item[R5] Data4Help is able to organize data provided by individuals.
\item[D3] Users have access to internet.



\end{enumerate}

\item[\textbf{[G2]}] \textbf{Third parties can have access to data of accepted data of groups requests.}

\begin{enumerate}

\item[D3] Users have access to internet.
\item[R1] Data4Help is able to store the data provided by individuals.
\item[R2] Users are able to login with the username and the password associated to their account.
\item[R5] Data4Help is able to organize data provided by individuals.
\item[R6] Third parties can formulate requests to access anonimized data of groups.
\item[R7] Third parties can apply filters on data while formulating their request for data of groups.
\item[R8] The system checks if the number of individuals in the group detected by the third party request is higher than 1000.
\item[R8.1] If the groups has 1000 or less individuals the system denies the request.
\item[R9] The system is able to distribute the requested data to the third party.



\end{enumerate}

\item[\textbf{[G3]}] \textbf{Third parties can have access to data of specific individuals that gave permission to it.}

\begin{enumerate}

\item[D3] Users have access to internet.
\item[D2] A third party interested in monitoring a specific individual knows the SSN of the individual.
\item[R1] Data4Help is able to store the data provided by individuals.
\item[R2] Users are able to login with the username and the password associated to their account.
\item[R5] Data4Help is able to organize data provided by individuals.
\item[R10] Data4Help is able to forward the request to the individual specified in the third party.
\item[R11] The individual to whom the request will be forwarded is able to reply.
\item[R12] The system notifies the third party about the individual reply.
\item[R13] Data4Help can check if the individual has accepted the data access request by the third party.
\item[R9] The system is able to distribute the requested data to the third party.


\end{enumerate}

\item[\textbf{[G4]}] \textbf{The system is able to notify third parties when new data of their observed groups or individuals is gathered.}

\begin{enumerate}
\item[D3] Users have access to internet.
\item[R2] Users are able to login with the username and the password associated to their account.
\item[R1] Data4Help is able to store the data provided by individuals.
\item[R5] Data4Help is able to organize data provided by individuals.
\item[R14] Third parties can specify to be updated whenever new data of groups detected by their data requests, or observerd individuals is gathered. The time interval between an update and the next one can be set by the third party, in a limit that goes from one hour to 1 year.
\item[R15] The system is able to update third parties with new data, respecting their prefecerences on the interval timing.
\end{enumerate}

\item[\textbf{[G5]}] \textbf{The system is able to start automatically a rescue procedure.}
\begin{enumerate}
\item[D3] Users have access to internet.
\item[R2] Users are able to login with the username and the password associated to their account.
\item[R3] Data4Help is able to pair with the user wearable.
\item[R16] Individual can subscribe to the AutomatedSOS service.
\item[R17] The individual wearable can measure at least two vital parameters.
\item[R18] The system is able to continuosly monitor individual vital parameters.
\item[R19] If the wearable is disconnected the AutomatedSOS service is suspended, untill the individual connects his wearable again.
\item[D4] Thresholds for health parameters are provided by medical experts.
\item[D5] An ambulance is always available when it is needed.
\item[D7] The ambulance driver can reach the user in critical condition.
\item[D1] Location and eHealth data are provided by individuals' devices and assumed to be correct.
\item[R20] Whenever a vital parameter cross its respective treshold, the system calls an ambulance in under 5 seconds, giving to the ambulance driver the location of the individual.
\end{enumerate}

\item[\textbf{[G6]}] \textbf{The system is able to mantain the desired privacy level of each individual.
}

\begin{enumerate}
\item[D1] Registered users must keep their login credentials secret.
\item[R21] The system doesn't allow third parties to access specific individuals data without first asking for their permission.
\item[R22] The system stops to update third parties about new data whenever observed individuals decide to remove data access permissions, or whenever the number of individuals of an observed groups goes belove 1001.
\end{enumerate}
\end{enumerate}


\subsubsection{Use case diagrams}



\begin{tabular}{|l|p{13cm}|}
    \hline
    Name & Sign Up
    \\ \hline
    Actor & Individual
    \\ \hline 
    Entry condition & The individual installed the application and is connected to the internet
    \\ \hline
    Events flow &
    \begin{enumerate}
    \item Open the application
    \item Click on "Register account"
    \item Fill all the mandatory fields with correct informations
    \item Click on "Confirm".
    \item The system correctly add individual credentials to the database
    \end{enumerate}
     \\ \hline
     Exit Condition & Individual credenials have successfully been added to the system,
     the individual can now log in     
     \\
    \hline
    Exceptions &
        \begin{enumerate}
    \item The individual username or his emai is already present in the system database.
    \item Individual didn't fill all the mandatory fields.
    \item Individual provided invalid data.
    \item All the exceptions above are notifieed to the individual who is taken back to the registraton form
    \end{enumerate}
      \\
    \hline
\end{tabular}



\begin{tabular}{|l|p{13cm}|}
    \hline
    Name & Sign In
    \\ \hline
    Actor & Individual
    \\ \hline 
    Entry condition & The individual has already signed up,has installed the application and is connected to the internet.
    \\ \hline
    Events flow &
    \begin{enumerate}
    \item Open the application
    \item Fill the username and password field in the app homepage.
    \item Click on "Log in".
    \item The username and the password inserted are sent to the system.
    \item The system checks if the credentials are 
    \item The individual is successfully logged in and the app redirect him to the wearable pairing screen.
    \end{enumerate}
     \\ \hline
     Exit Condition & Individual visualizes the wearable pairing screen
     \\
    \hline
    Exceptions &
        \begin{enumerate}
    \item The inserted credentials are not found in the system database. 
    \item The application notifies that the inserted credentials are wrong.
    \end{enumerate}
      \\
    \hline
\end{tabular}


\begin{tabular}{|l|p{13cm}|}
    \hline
    Name & Wearable pairing
    \\ \hline
    Actor & Individual
    \\ \hline 
    Entry condition & The individual has logged in and both his wearable and his phone have bluetooth on.
    \\ \hline
    Events flow &
    \begin{enumerate}
    \item The individual reaches the wearable pairing screen.
    \item The individual taps on pair wearable.
    \item The individual ensures that the wearable is on and close to the smartphone.
    \item The application successfully recognize the wearable.
    \end{enumerate}
     \\ \hline
     Exit Condition & Individual wearable is paired and the app is ready to gather individual eHealth data through it.
     \\
    \hline
    Exceptions &
        \begin{enumerate}
    \item The application isn't able to recognize a wearable. 
    \item The exception above is notified by the application. The individual can repeat the pairing procedure.
    \end{enumerate}
      \\
    \hline
    Special Requirements & The pairing time limit is 10 seconds.
\end{tabular}



\begin{tabular}{|l|p{13cm}|}
    \hline
    Name & Upload Data
    \\ \hline
    Actor & Individual
    \\ \hline 
    Entry condition & The individual is logged in and the wearable is paired.
    \\ \hline
    Events flow &
    \begin{enumerate}
    \item The wearable has gathered some individual eHealth data.
    \item The application download the data from the wearable.
    \item The application uploads the data on the system.
    \end{enumerate}
     \\ \hline
     Exit Condition & Individual eHealth data is successfully uploaded and stored in the system
     \\
    \hline
    Exceptions &
        \begin{enumerate}
    \item Connection falls down during the upload phase.
    \item The application fails the download of eHealth data from the individual's wearable.
    \item The application notifies that there were problem during the upload phase. The application will automatically 
    \end{enumerate}
      \\
    \hline
    Special Requirements & EHealth data upload should take less than 500 milliseconds.
\end{tabular}



\begin{tabular}{|l|p{13cm}|}
    \hline
    Name & Manage third parties data access requests.
    \\ \hline
    Actor & Individual, Third party
    \\ \hline 
    Entry condition & The individual is logged in and a data access request notification from a third party is received.
    \\ \hline
    Events flow &
    \begin{enumerate}
	\item The individual goes to the third parties access app section.
    \item The individual answers the data access request, giving or not giving the permission to the third party to access his data.
    \item The application forward the individual response to third party.
    \end{enumerate}
     \\ \hline
     Exit Condition & The third party is correctly notified with individual decision
     \\
    \hline
    Exceptions &
        \begin{enumerate}
    \item Connection falls down while the individual is answering the request.
    \item The application notifies that the device is disconnected and temporally store offline the individual answer. The answer will be then uploaded as soon as the connection comes back
    \end{enumerate}
      \\
    \hline
\end{tabular}



\begin{tabular}{|l|p{13cm}|}
    \hline
    Name & Manage third parties data access
    \\ \hline
    Actor & Individual
    \\ \hline 
    Entry condition & The individual is logged in and wants to remove a third party data access.
        \\ \hline
    Events flow &
    \begin{enumerate}
	\item The individual goes to the third parties access app section.
    \item The individual selects the third party he wants to remove.
    \item The individual clicks on the remove button.
    \end{enumerate}
     \\ \hline
     Exit Condition & The system correctly change data access permissions and notifies the third party of the change
     \\
    \hline
    Exceptions &
        \begin{enumerate}
    \item No third party already has no access to individual's data.
    \item The exception above is handled displaying "No third party has access to your data"
    \item The connection goes down during the process. 
    \item The exception above is handled notifying the individual that there is no connection. The individual must repeat the whole procedure when the connection is back.
    \end{enumerate}
       \\
    \hline
\end{tabular}



\begin{tabular}{|l|p{13cm}|}
    \hline
    Name & Subscribe to AutomatedSOS service
    \\ \hline
    Actor & Individual
    \\ \hline 
    Entry condition & The individual is logged in and the wearable is paired.
        \\ \hline
    Events flow &
    \begin{enumerate}
	\item The individual goes to the AutomatedSOS app section.
    \item The individual click the subscribe button.
    \item The app checks if the wearable paired can gather the necesessary data for the AutomatedSOS service.
    \item The app starts monitoring the indivudal.
    \end{enumerate}
     \\ \hline
     Exit Condition & The wearable starts monitoring individuals vital parameters.
     \\
    \hline
    Exceptions &
        \begin{enumerate}
    \item Wearable cannot monitors individuals vital parameters.
    \item The exception above is notified to the individual.
    \item Wearable unpairs during the procedure.
    \item The app notifies the exception above to the individual.The individual must click on the subscribe button again.
    \item The exception above is handled notifying the individual that there is no connection. The individual must repeat the whole procedure when the connection is back.
    \end{enumerate}
      \\
    \hline
\end{tabular}



\begin{tabular}{|l|p{13cm}|}
    \hline
    Name & Ambulance automatic call
    \\ \hline
    Actor & Individual, Ambulance
    \\ \hline 
    Entry condition & One of the individual's vital parameters cross its critical treshold.
        \\ \hline
    Events flow &
    \begin{enumerate}
    \item The application uses the individual's phone GPS to find his position
	\item The application starts an automatic call to the emergency line.
	\item The applcation gives the route and forward individual vital parameters values to the emergency line.
    \end{enumerate}
     \\ \hline
     Exit Condition & The ambulance is dispatched.
     \\
    \hline
    Exceptions &
        \begin{enumerate}
    \item Wearable unpairs during the procedure.
    \item The procedure continues with the last saved location and vital parameters.
    \end{enumerate}
      \\
      \hline
    Special Requirements & The system must call the ambulance in under 5 seconds from the treshold crossing.
      \\
      \hline
\end{tabular}



\begin{tabular}{|l|p{13cm}|}
    \hline
    Name & Individual data request
    \\ \hline
    Actor & Third party
    \\ \hline 
    Entry condition & The third party is logged in and knows the individual's username or tax code
        \\ \hline
    Events flow &
    \begin{enumerate}
    \item The third party selects the data request button
	\item The third party insert the individual identifier (either his username or his tax code)
	\item The system forwards the request to the individual.
    \end{enumerate}
     \\ \hline
     Exit Condition & A notification of a new data access request is sento to the individual.
     \\
    \hline
    Exceptions &
        \begin{enumerate}
    \item The individual identifier is not found in the system.
    \item The exception above is notified to the third party, wich is sent back to the data request screen
    \end{enumerate}
        \\  \hline

\end{tabular}



\begin{tabular}{|l|p{13cm}|}
    \hline
    Name & Group data request
    \\ \hline
    Actor & Third party
    \\ \hline 
    Entry condition & The third party is logged in
        \\ \hline
    Events flow &
    \begin{enumerate}
    \item The third party goes in the data request section.
    \item The third party selects the group data request
	\item The third party can apply filters on his request. To be precise it can filter data by individuals age, sex, location (country, city) and can specify wich data it is interested in (steps taken daily, daily sleep hours, average heart rate, average blood pressure)
	\item The system analyzes the request and distributes the requested data to the third party.
    \end{enumerate}
     \\ \hline
     Exit Condition & The requested data is visualized by the third party.
     \\
    \hline
    Exceptions &
        \begin{enumerate}
    \item The group of individuals detected by the third party has a number of members of 1000 or less.
    \item The exception above is notified to the third party, wich is sent back to the data request screen.
     
    \end{enumerate}
	 \\
    \hline
\end{tabular}



\begin{tabular}{|l|p{13cm}|}
    \hline
    Name & Data visualization
    \\ \hline
    Actor & Third party
    \\ \hline 
    Entry condition & The third party is logged in.
        \\ \hline
    Events flow &
    \begin{enumerate}
    \item The third party goes in the data visualization section.
    \item The third party is able to see
	\item The third party can apply filters on his request. To be precise it can filter data by individuals age, sex, location (country, city) and can specify wich data it is interested in (steps taken daily, daily sleep hours, average heart rate, average blood pressure)
	\item The system analyzes the request and distributes the requested data to the third party.
    \end{enumerate}
     \\ \hline
     Exit Condition & The requested data is visualized by the third party.
     \\
    \hline
    Exceptions &
        \begin{enumerate}
    \item The group of individuals detected by the third party has a number of members of 1000 or less.
    \item The exception above is notified to the third party, wich is sent back to the data request screen.
    \end{enumerate} 
     \\
    \hline
\end{tabular}
  

   
\begin{tabular}{|l|p{13cm}|}
    \hline
    Name & Data visualization
    \\ \hline
    Actor & Third party
    \\ \hline 
    Entry condition & The third party is logged in and has access to some data (either group or individual)
        \\ \hline
    Events flow &
    \begin{enumerate}
    \item The third party goes in the data visualization section.
    \item The third party click on the data it is interested to subscribe to.
	\item The third party clicks on the subscribe button.
	\item The system starts automatically updating the data to wich the third party is subscribed to.
    \end{enumerate}
     \\ \hline
     Exit Condition & The application notifies the third party of the new subscription.
     \\
    \hline
    Exceptions &
        \begin{enumerate}
    \item The number of individuals of the group detected by the data request wich the third party wants to subscribe to went belove 1001 members.
    \item The individual removed the permission to access his data from the third party.
        \item The exceptions above are notified to the third party, wich is sent back to the data visualization screen.
    \end{enumerate}   
  \\
    \hline
\end{tabular}







\subsubsection{Sequence diagrams}

\section{Performance Requirements}
Data4Helps is a service that strongly depends on the amount of data gathered therefore the initial phase will focus on recruiting at least 100000 individuals while the number of third parties using the service in the initial phase it's expected to be around 500. In order to expand the user base and have from the start good feedbacks third parties data requests should have a response time of 1 second or less while the upload time of individuals data should be less than 500 milliseconds. Obviously these parameters are flexible, and under heavy loads may be crossed.Having a robust, persistent and fast database is of crucial importance.

\section{Design constraints}
\subsection{Standards compliance}
\begin{itemize}
\item The app only requests the minimum permissions in order to guarantee its core functionalities: storage access, location, access to wearable.
\item The app only supports portrait mode.
\item All the data gathered trhough individuals wearables will be uploaded and stored in the system database.
Critical treshold for vital parameters will be stored in the user device in order to guarantee the AutomatedSOS even when the phone is offline.
\end{itemize}


\subsection{other costraints}

\section{Software system costraints}
\subsection{Reliability}
For this tipology of application is necesessary to guarantee a 24/7 service.
\subsection{Availability}
The system is expected to be available 99.99% of the time.
\subsection{Security}
Users passwords will be cripted and stored in a specific database.
\subsection{Mantainability}
The whole system will be designed in a modular way in order to separate each component from the other ones, guaranteing an high level of flexibility. AutomateSOS is a first, clear example of the expandability nature of Data4Help that will both analyze individual and third parties tips to improve its service.
\subsection{Compatibility}
Gathering the possible largest amount of eHealth data is the fundamental feature of the service. Although, due the fragmentation of the wearable market, only watchOS and wearOS smartwatches will be initially supported. TizenOS and wearOS smartwatches will get official support as soon as possible, not longer than 3 months after the app initial release.
The app will support both Android and iOS smartphones. 