\chapter{Overall description}
\section{Product perspective}
Data4Help is a brand new service that will be built from the ground up. The user will interact with the system through the mobile application. An internet connection, GPS, and a wearable to collect eHealth data are required in order to use Data4Help fuctionalities and the optional additional service AutomatedSOS.

The following class diagram will be helpful to understand how data4hel will help to handle both individuals and third parties functionalities.

\begin{figure}[H]
  \includegraphics[width=0.79\linewidth]{resources/UML/Data4HelpClassDiagram.png}
  \caption{Data4Help Class Diagram}
  \label{fig: Data4Help Class diagram}
\end{figure}

The following statechart diagram briefly overview how the AutomatedSOS service works:

\begin{figure}[H]
  \includegraphics[width=0.79\linewidth]{resources/UML/AutomatedSOSstatechart.png}
  \caption{AutomatedSOS statechart diagram}
  \label{fig: AutoamtedSOS statechart diagram}
\end{figure}


\subsection{Access to specific data}
A third party interested in monitoring a specific individual can send a request to the system specifying his SSN or TC.
The system passes the request to the specific individual who can accept or refuse it.
If the request is accepted, the third party can see daily or aggregate data of the individual. In this case the third party can also see name, surname and age of the individuals.


\subsection{Access to group data}
A third party can also request access to anonymized data of a group of individuals.
To do so, it must specify some parameters concerning individuals. These parameters can regard geographical area, age range or level of education of the users of interest to the third party.
Geographical area can be specified in term of country, region, province, town and district (only for big city).

This type of requests are managed directly by the system. Because TrackMe holds in high regard the privacy of its users, it will satisfy the request only if the number of individuals whose data satisfy the request is higher than 1000.

If the request is positively evaluated, the anonymized data is made available to the third party.
Optionally, the third party can subscribe to new data with the same characteristics. If new data, matching the third party request, is produced it will be made available to the requester at most once a week.




\subsection{SOS service}
The service AutomatedSOS must exploit the real-time stream of eHealth data provided by the underlying Data4Help to offer a personalized and non-intrusive SOS service especially designed for elderly people.
Any individual, correctly registered to Data4Help, can optionally request to active this service


If AutomatedSOS has been activated, the system should continuously monitor the parameters of the individuals and compare them using certain specific thresholds.
If the parameters are below these thresholds, the system assumes that the individual is having a sudden illness and forwards a request to an emergency service for sending an ambulance to the location of the user.
AutomatedSOS must ensure a reaction time of less than 5 seconds from the time the parameters are detected below the threshold.










\section{User characteristics}
The users of Data4Help and AutomatedSOS services are:

\begin{itemize}
\item \textit{Individual} : user that allows the acquisition of the data and can optionally activate AutomatedSOS. He can't use the data request feature of Data4Help.
\item \textit{Third party}: user that can request data from the application.
\end{itemize}





\section{Assumptions, dependencies and constraints}

\subsection{Domain assumptions}

\begin{itemize}


\item[] \dom{ 
Location and eHealth data are provided by individuals' devices and assumed to be correct.
}{ProvidingData} 
\item[]\dom{
A third party interested in monitoring a specific individual knows the SSN of the individual
}{SSNknown}
\item[]\dom{
Users have access to internet.
}{Internet}
\item[]\dom{
Thresholds for health parameters are provided by medical experts.
}{Tresholds}
\item[]\dom{
An ambulance is always available when it is needed.
}{Ambulance}
\item[]\dom{
Registered users must keep their login credentials secret.
}{Credential}
\item[]\dom{
The ambulance driver can reach the user in critical condition.
}{Reachability}
\end{itemize}




\subsection{Privacy constraints}
The system will collect and elaborate personal data of the individuals and, possibly, it will share them or part of them with third party.  For this reason, during the registration activity to the system, all the users must be informed of this practice and they must explicitly confirm their consensus.

In particular, anonymized data can be shared with third parties who request it without the users being further advised. In order to protect its users' privacy and to prevent misuse of data, TrackMe won't share data if the number of individuals whose data satisfy the request is lower than 1000.

Moreover, a third party can request to fully access the data of some specific individual. In this case it is up to the individual to accept or not to share his data with that specific third party.

\subsection{Hardware limitations}
Individuals will use a smartphone app in order to acess the service.
The following devices are the one that will be compatible with the app:
\begin{itemize}
\item iOS or Android smartphone with following capabilities:
\begin{itemize}
\item 2G/3G/4G connection
\item Bluetooth connection
\item GPS
\end{itemize}
These are instead the smartwatches that will be initially supported in the app:
\item watchOS or wearOS smartwatch
\newline
Third parties will have access to the service through a WebApp, accessible with any browser compatible with html5 and java.
\end{itemize}

