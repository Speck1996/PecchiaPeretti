\section{Installation instructions}
The following software are needed

\begin{itemize}
\item MySql Community Server  5.7.24 or MySql Community Server 8.0.13
\item MySql Connector/J 5.1.47
\item MySql Workbench 8.0.13 (Optional)
\item JDK 8 Update 144
\item Glassfish 5.0
\end{itemize}




\subsection{Database initialization}
After having installed MySql Server, you have to create the database for trackme.
In the repository you can find 3 sql file to do this:

\begin{enumerate}
\item \texttt{create.sql}
\item \texttt{sample.sql}
\item \texttt{bigsample.sql}
\end{enumerate}
As their name suggest, the first one create the database and the tables, the second populated the tables with some small data, while the last one insert a larger, randomly generated, amount of data.
Notice that running any of these files will drop the existing data currently stored in the database.
\vspace{1em}

\noindent 
In order to execute those sql you can type in the terminal

\begin{center}
\texttt{mysql -u your\_username -p \textless\ you\_file.sql}
\end{center}
and enter your password.
Alternatively, you can use MySql Workbench with \textit{File/Run sql script}.


\subsection{Connection pool configuration}
Assuming \textit{domain1} is in use on glassfish, copy \texttt{mysql-connector-java-5.1.47-bin.jar} in the folder

\begin{center}
glassfish5/glassfish/domains/domain1/lib 
\end{center}
Now follow the next steps.
\begin{enumerate}
\item Start glassfish server.
\item Open the administration panel (usually on port 4848).
\item Select \textit{Resources/JDBC/JDBC Connection Pools} on the tree on the left.
\item Click \textit{New..}.
\item In the \textit{Pool Name} field enter \texttt{TrackmePool}.
\item As \textit{Resource Type} select \texttt{javax.sql.DataSource}.
\item As \textit{Database Driver Vendor} select \texttt{MySql}.
\item Click \textit{Next}.
\item Make sure that \textit{Data Source Classname} is \texttt{com.mysql.jdbc.jdbc2.optional.MysqlDataSource}.
\item In \textit{Additional Properties}, you can delete all the default ones and set the following:
	\begin{itemize}
	\item \texttt{portNumber}: the port of your mysql server, usually \texttt{3306};
	\item \texttt{user}: your username for the mysql server;
	\item	\texttt{password}: your password for the mysql server;
	\item \texttt{databaseName}: enter the value \texttt{trackme} as defined in the SQL DD.
	\end{itemize}
\item Click \texttt{Finish}.
\end{enumerate}

Now your database connection pool should be configured, you can check selecting the connection pool and clicking \textit{Ping} (be sure that the MySql server is running).
A \textit{Ping Succeeded} message should show up.\vspace{1em}


\noindent
In order to allow the application to use this connection pool, a new JNDI resource must be created.
From the glassfish administration panel:

\begin{enumerate}
\item Select \textit{Resources/JDBC/JDBC Resources} on the tree on the left.
\item Click \textit{New..}.
\item As \textit{JNDI Name} enter \texttt{jdbc/trackme} (note that this name is important, if different the application will not be able to locate the resource).
\item As \textit{Pool Name} select the \texttt{TrackmePool} just created.
\item Click \textit{OK}.
\end{enumerate}


\subsection{Certificate issue for Nominatim API}
It can happen that the call the to Nominatim API fails due to certification issue.
This problem can be noted in the Glassfish log file when trying to perform a call for a location coordinates.
If the problem is not fixed, the location constraints for group data request are simply ignored.
\vspace{1em}

\noindent
A possible fix is to use the certificates distributed along with the JDK 8.
From the installation folder of the JDK, go to Contents/Home/jre/lib/security and copy the \texttt{cacerts} file.
Go to the Glassfish installaton folder and then, assuming domain1 is in use, to glassfish/domains/domain1/config.
Here there should be an already exiting \texttt{cacerts.jks}, replace this file with the previously copied from the JDK (adding the extension could be necessary).




\subsection{Web App Deployment}
After downloaded the \texttt{trackme.war} from the repository, follow the next steps in order to deploy the application. Be sure that the mysql server is running and the connection pool properly configured.

\begin{enumerate}
\item Start glassfish server.
\item Open the administration panel (usually on port 4848).
\item Select \textit{Applications} on the tree on the left.
\item If not already selected, select \textit{Packaged File to Be Uploaded to the Server} and choose the \texttt{trackme.war} previously downloaded.
\item In the \textit{Contex Root} field enter \texttt{trackme}.
\item Click \textit{OK}.
\end{enumerate}
Now the trackme server should have been correctly deployed, to verify this, go to
\begin{center} http://localhost:8080/trackme/login.xhtml\end{center}
the login page (or the homepage if the browser sent a valid authentication cookie) should appear.

