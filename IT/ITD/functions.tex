\section{Requirements Implemented}

The developing of the system started with the design and  the implementation of the DB. As stated in the DD, the server is considered as the main component in whole system. This is because it handles both the controller and the model in the MVC pattern used to build Data4Help.
This also is coherent with what stated in the RASD, where Data4Help is described as a central hub for eHealth data.
After finishing the DB, the development of the RequestManager component started. This component handles all the necessary operations to distribute and perform data requests to the DB. In particular in addition to the data retrieving it checks for privacy constraints for both individual and group requests, and it also is in charge to transform location string filters in coordinates, using an external service: Nominatim.
Then the next item to be implemented was the DataCollector: this component is in charge of taking individual's data and inserting it into the DB.
After having the eHealth data handling components ready the development of the login service started. The login service handles the login process for the WebApp and the Android App by using tokens. Tokens are generated by the server and stored by the clients, that have to provide the token each time they want to access a resource that requires authentication.
Once the login service was ready the android app implementation started, from the login itself. 
Some of the app features provided in the RASD were not implemented: for example there is no way to simulate sensors eHealth data from emulated smartwatches and neither one of the two team member has an android wearOS smartwatch. During the development there was an attempt to gather the data from google fit servers using Google Fit History API, but since Google itself states in the documentation that the data dathered is not reliable, the feature was dropped for impossibility of testing it. This lead to the decision of not implementing the data history visualization process for AndroidApp, as it was mainly focused on keeping real data coming from smartwatches on Data4Help DB, so that individuals could have a backup of their eHealth Data indipendently from the owned smartwatch.
The data gathered indeed is just some random data distributed with a gaussian, with its parameter settable from a proper screen.
Then the requests are fully functioning with the user that can accept or refuse requests.
In the end the WebApp was developed in all his functionalities allowing third parties to make and visualize their requested data.


The following list presents implemented requirements from the RASD with a brief description on where to find the source code that realizes them:

\begin{itemize}
\item Users are able to login with the username and the password associated to
their account: The code that implement this feature can be found in the login package.
\item Individuals can upload their data through Data4Help app: The code of this feature can be found in MyDataActivity Class of the AndroidApp. Since the smartwatch pairing was not implemented, the data is randomly generated.
\item Data4Help is able to store the data provided by individuals: This is one of the most important features of Data4Help, and it is implemented in the collector package by the DataCollecor, with the help of a MySql DB.
\item Data4Help is able to organize data provided by individuals: This feature is implemented in the model package where each kind of data has is own class, that corresponds to a table in the DB.
\item Third parties can formulate requests to access anonymized data of groups: This feature is implemented by the WebApp with the help of EJB that can be found in the webapp package.
\item Third parties can apply filters on data while formulating their request for
data of groups: Feature implemented in the WebApp.
\item The system checks if the number of individuals in the group detected by
the third party request is higher than 1000: This requirement is implemented by the GroupRequestManager class, in the manager package.
\item If the groups has 1000 or less individuals the system denies the request: Feature implemented in the GroupRequestManager.
\item The system is able to distribute the requested data to the third party: This functionality is provided by the WebApp with the help of EJB.
\item Third parties can formulate requests to access specic individual data,
indicating the individual's TC: This feature is implemented by the IndividualRequestManager, in the manager package.
\item Third parties can apply filters on the type of data while formulating their
request for individual data: This functionality is implemented by the WebApp.
\item Data4Help is able to forward the request to the individual specified by
the third party: The code of this feature can be found in the RequestFragment in the AndroidApp and in the individual request manager for what regards the server part.
\item The individual to whom the request will be forwarded is able to reply: Implemented in the RequestFragment of the AndroidApp
\item The system notifies the third party about the individual reply: Feature implemented by the WebApp: the WebApp doesn't have a real notification system, but just displays new data when it is available.
\item The system can check if the individual has accepted the data access
request by the third party: Feature implemented in the IndividualRequestManager through the monitoring table in the DB.
\item The system is able to update third parties with new data, respecting their
preferences on the interval timing: Feature implemented by the WebApp, where the third party can choose the update frequency
\item The system doesn't allow third parties to access specific individuals data
without first asking for their permission: Functionality implemented by the IndividualRequestManager
\item The system stops to update third parties about new data whenever observed individuals decide to remove data access permissions, or whenever the
number of individuals of an observed groups goes belove 1001: Feature implemented by the IndividualRequestManager.
\item Only users that know their username and password can access to their
respective accounts: Functionality implemented in the login package.
\end{itemize}


The following list instead contains the functionalities that were not implemented, or that were dropped during the development phase:

\begin{itemize}
\item Data4Help is able to pair with the user wearable: This requirement was not realized, because emulated smartwatches can't emulate some sensors like the heart rate sensor, making the pairing useless since no eHealth data can't be retrieved from emulated smartwatches.
\item Data4Help is able to download eHealth data from the individual's wearable: As for the pairing requirement, this feature can't be implemented with emulated smartwatches.
\end{itemize}


The other requirements that are not present in these lists are part of the AutomatedSOS service, whose implementation was not required.