\section{Structure of the source code}
The source code is clearly divided in two parts: the part that implements the TrackMe server and the part that implements the Android application.


\subsection{Web server and Web App}

The source code of the server side is subdivided in the following packages: \texttt{collector, login, manager, model, webapp}.

\paragraph{Collector}
In the \texttt{collector} package is located the fundamental \texttt{DataCollector} class whose task is, basically, to receive the data from the Android application and persist it in the database.

\paragraph{Login}
In the login package are located all the classes that implements functionality related to login, signup and authorization.

\paragraph{Manager}
In the \texttt{manager} package is located the logic that manages the request, forwarded by third parties by means of the web app, and the visualization of data.
It also contains a sub package, called \texttt{geocoding}, in which are present interfaces and classes responsible for the maps external API.

\paragraph{Model}
The \texttt{model} package contains all the entities used in the Object-Relation Mapping, exploited by means of JPA specification.
It contains also the definition of the enumeration and constant used in the database.
In the sub package called \texttt{xml} are located some classes useful for marshalling data into xml.

\paragraph{Webapp}
In the \texttt{webapp} package are located the managed beans that constitute the back-end of the webapp (excluded those regarding login and signup that are located in the login package).

\vspace{1em}
\noindent
Besides the Java source code, the server also contains, in the \texttt{web} folder, all the front-end of the webapp developed using JSF alongside to some CSS.


\subsection{Android}

The source code of the AndroidApp is divided in the following packages \texttt{activities, model, retrofitclient, adapter, helpers}.

\paragraph{Activities}
In this package all the classes that handles view operations and rendering can be found. In particular to each activity or fragment corresponds a screen UI of the Android App.

\paragraph{Model}
In the \texttt{model} package there are the classes needed to communicate with Data4Help server.

\paragraph{Retrofitclient}
In the package \texttt{retrofitclient} there are the interface used to communicate with the server, and the class in charge of building the RetrofitClient.

\paragraph{adapter}
In the \texttt{adapter} package there is the class in charge of rendering the requests view, that are then used in the RequestFragment that can be found in the \texttt{activities}.

\paragraph{helpers}
In the \texttt{helpers} package there are utilities needed by the activities: an encryptor used to encrypt passwords in the login and signup activities and a DatePicker used to have a better view experience in the signup for what regards date input.


